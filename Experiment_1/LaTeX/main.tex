\documentclass[prX,nofootinbib,notitlepage,12pt]{revtex4-1}

% For two-column format (which usually looks more professional):
%\documentclass[twocolumn,prX,nofootinbib,notitlepage]{revtex4-1}

% Note: The revtex4-1 document class includes a variety of Physics Journal
% styles. The option prX is specifies the Physical Review X journal style.


%*************************** Load Packages *******************************
% Note: Package descriptions and usage documentation can be found online
% at https://ctan.org/. Just use the site's search bar to search for the
% package

%------------------ Extending default LaTeX Structures -------------------
\usepackage{enumerate}
\usepackage{float}
\usepackage{caption}
\usepackage{hyperref}
\usepackage{subfiles}
\usepackage{enumitem}

%------------------------ Symbology \& Math ------------------------------
\usepackage{amsmath, amsthm, amssymb, amsfonts}
\usepackage{mathrsfs}
\usepackage{array}

%--------------------------- Graphics ------------------------------------
\usepackage[dvips]{graphics}
\usepackage{graphicx}
\usepackage{color}


%*************************************************************************
%****************************** Body *************************************
%*************************************************************************
\begin{document}

%************************ Document Settings ******************************

%---------------------- Bibliography Settings ----------------------------
\bibliographystyle{plain}


%****************** Title Information \& Abstract ************************

%--------------------------- Title ---------------------------------------
\title{PHYS 506 - Experiment 1 Semi-Report}

%----------------------- Author Information ------------------------------
\author{Luke Abanilla}%
\author{Eben Quenneville}
\author{Augustus Vigorito}
\affiliation{Department of Physics, University of New Hampshire,
Durham, NH 03824, USA}

\maketitle

\section{Design Considerations}

\begin{enumerate}[label=(\alph*)]
    \item \textbf{Consider the RSpec Explorer software you used in Part 1. Suppose you are a software
developer working on improving this software for academic/teaching or for experimental/research purposes. What other useful features could you add to the software to
make it easier to identify unknown gases from their emission spectra? (Assume you
have any/all relevant computational skills to implement whatever helpful features you
want to add.)}

We would have the software automatically set the origin light source such that we would not have to always recalibrate the software for every small minute change in the distance or a minor adjustment in the angle to the light source. As for the data collection, we thought it would be useful to be able to directly edit the units and scale of the graphs.

\

    \item \textbf{How might you change the experimental setup or environment in Part 2 to help better
identify the unknown gas?}


For part 2 of the experiment we would want to attach the light source and the spectroscope together such that the position and angle of the light source is not changed by error. Another thing we would have change is the focal length or at least of the ability to adjust focus such that it would be much easier to observe the spectral lines with little ambiguity.

\end{enumerate}

\pagebreak

\section{Historical Notes}

\textbf{In your own words, discuss the historical aspects of the experiment from Part 2, as if you
were writing a Historical Background section of a report or scientific article.}
Some things worth considering:
\begin{enumerate}
    \item What were the historical and scientific consequences of the original spectral lines experiments?
    \item Who first studied spectral lines?
    \item How has physics changed because of this then-newfound understanding of spectral lines?
\end{enumerate}
\textbf{\textit{(It will be helpful to refer to your Modern Physics textbook, or other resources, to help you
write a few paragraphs to answer these questions. This may require a little bit of research.)}}

\begin{enumerate}
    \item Following the founding of spectral analysis by Gustav Robert Kirchoff and Robert Bunsen, scientists were able to study the composition of matter and identify elements based on their spectral characteristics
    \item The first study on spectral lines originated from Isaac Newton who created an apparatus that consisted of an apreture, lens, prism, and screen to create the first spectroscope to analyze the series of colors generated from white light from the sun.
    \item After the discovery of spectral lines, scientists were able to use it for identification of matter or elements which led to furtherer discoveries such as solar spectral lines, radioactivity, quantum mechanics, wave mechanics, and atomic absorption spectroscopy.
\end{enumerate}


\pagebreak
\section{Theoretical Background}

\textbf{In your own words, discuss the theoretical background of the experiment from Part 2, as if
you were writing a Theoretical Background section of a report or scientific article.
Consider answering and discussing the following questions:}
\begin{enumerate}[label=(\alph*)]
\item How is light emitted by the gas tube lamp? What physical mechanisms come into play
here?
\item How does a difraction grating separate incident light waves into spectra?
\end{enumerate}

\textbf{Answering these questions may include the use of (and/or derivation of) equations.}


In this lab, we used an apparatus that placed a gas tube between two terminals with a large voltage difference between them. The voltage excites the atoms of the gas to a higher energy state (from $n = 0$ to $n = 1, 2, \dots$.) These higher energy levels are not stable, however. They eventually decay, and the electrons return to the ground state. When the electrons drop back down to a lower energy level, they emit photons at a specific frequency determined by the drop in energy as governed by the equation

$$
    \Delta E = \frac{hc}{\lambda}
$$

The distribution of various energy and excitement levels among the atoms of a given species give rise to a characteristic atomic spectrum of emissions. That is, each element has a certain combination of possible energy states. When the atoms move between them they release or absorb photons of a unique but predictable distribution of frequencies. We can decompose that spectrum of emissions for a given element in the gas tube using diffraction grating. A diffraction grating is essentially a fine mesh with thousands of tiny slits in it at specific intervals. The slits allow us to divide each wavelength of light at a different angle. Each wavelength can then be focused onto a photo-detector to analyze the data using spectroscopy, as in Part 1, or to be viewed by the human eye, as in Part 2 of this experiment.

\end{document}